%\begin{itemize}
%\item Interactive IBR faces challenge of uncertainty in badly reconstructed and unreconstructed regions
%\item Ad-hoc methods, uncertainty not treated in a principled manner, depth synthesis single view, heuristic
%\item Two classes of IBR, forward efficient and high quality (warp), backprojection, handle uncertainty adhoc low quality or are slow.
%\end{itemize}

Recent interactive image-based rendering (IBR) methods allow convincing free-viewpoint navigation
in scenes reconstructed from a small number of images \cite{Eisemann08FT,dibr,chaurasia13,ODD15}.
Despite impressive advances in recent years, these methods are limited in
regions of the scene, which are badly or completely unreconstructed.
Such regions have varying degrees of \emph{uncertainty}, which 
previous solutions treat with heuristic methods: e.g., using dense matching~\cite{dibr}
or heuristic depth synthesis~\cite{chaurasia13}. 
In this paper we propose a principled approach to model uncertainty in IBR, coupled
with a depth synthesis and a IBR algorithm which build on this model.

The idea of modelling uncertainty goes back to the early days of
computer vision~\cite{Szeliski}, but tractable and efficient solutions
have not been developed for interactive IBR. Oversegmentation-based IBR methods achieve
good performance and quality by \emph{forward warping} regions of the image \cite{Zitnick:2004:viewinterp,chaurasia13},
but do not include a measure of uncertainty. Uncertainty has recently
been modeled for more traditional depth-based IBR algorithms \cite{devernay},
which can be considered as derivatives of the Unstructured Lumigraph~\cite{ULR} and use
\emph{backprojection} into the input images. However, the cost of this model is high
and the quality is not always satisfactory, in part because these methods do not use
oversegmentation to preserve depth boundaries.

Our solution addresses these shortcomings by developing a comprehensive model
of uncertainty for IBR. We identify all sources of uncertainty throughout the various processing
stages of modern IBR (calibration, reconstruction, depth synthesis etc.).
We use this model to introduce an iterative multi-view depth synthesis algorithm,
with a Bayesian approach to minimize uncertainty. This preprocessing approach is driven by evaluation of
IBR image quality, and allows fusion of multiple sources of 3D information.
The resulting model of uncertainty 
allows us to develop an IBR algorithm that combines the advantages of  
oversegmentation based forward warping and principled ULR-style backprojection,
notably allowing the use of information from more cameras while maintaining the performance
advantages of forward warping. Finally, we show how to provide plausible 
stereoscopic rendering for regions such as vegetation which can be considered volumetric.

Our contributions are thus:
\begin{itemize}
\item A comprehensive model of uncertainty for interactive IBR.
\item A principled iterative multi-view depth synthesis algorithm, which builds on and informs the uncertainty model.
\item A unified IBR algorithm, which provides a good quality/speed tradeoff by combining the advantages of forward warping and depth-based backprojection algorithms and includes plausible stereoscopic rendering for unreconstructed volumetric regions.
\end{itemize}
Our results show significant improvement over recent algorithms for scenes with poor reconstruction,
allowing the user to move much further from the input viewpoints with the same number of input images.

