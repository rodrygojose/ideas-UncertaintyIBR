The IBR processing involves several steps: camera calibration, reconstruction, oversegmentation 
and depth synthesis. 
Each of these steps treats and generates \emph{uncertain data}, possibly coming from multiple sources.
Our goal is to define a model for this uncertainty that will be used to provide a principled
IBR algorithm.

In this work, we chose to consider the first two purely computer vision steps,
i.e., camera calibration and reconstruction, as given. Our model will build on the
data these steps provide, but will not affect these algorithms. The main reason
for this is that calibration and reconstruction often have goals other than image
rendering, e.g., modelling, measurements, fabrication etc. In contrast all subsequent
steps for IBR, e.g., depth synthesis \cite{chaurasia13}, or Bayesian algorithm
labels \cite{ODD15} are specifically targeted to rendering. Consequently
we propose new solutions for these steps, which are tightly integrated with
our uncertainty model.

We will model uncertainty of various quantities used in IBR, namely 3D reconstructed points,
superpixels\cite{slic12}, synthesized depth \cite{chaurasia13} and estimated planes ~\cite{ODD15},
including the influence of each quantity on each other.
\begin{comment}
Or a abstract model that includes the effect of all uncertainties. 
\end{comment}

For 3D reconstructed points we will use the estimation of an ``uncertainty volume''
around the reconstructed parts of the scene, following~\cite{devernay}.
\TODO{give some detail}

For superpixels, we will evaluate how well the oversegmentation algorithm has captured
depth. Even though this oversegmentation is a general approach, our quality criteria
for superpixels are specifically guided by rendering quality. 
\TODO{Possible measures of quality involve uniformity of depth in a superpixel,
some measure of compactness. Other ?}
 
For depth synthesis and plane estimation quality, we will then use ``leave-one-out'' 
quality evaluation~\cite{ODD15}, during
our iterative multi-view depth synthesis (see Sec.~\ref{sec:synth}).
We define the model here, noting the quantities which are iteratively
updated during depth synthesis. \TODO{do this}

The representation is compact and efficient. \TODO{Will be developed iteratively as
the method becomes more precise.}

%\begin{itemize}
%\item reconstruction *(Pujades)
%\item depth synthesis
%\item depth resynthesis
%\item plane estimation
%\item compact and efficuient representation
%\end{itemize}

